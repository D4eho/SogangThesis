% !TEX encoding = UTF-8

\documentclass[11pt,a4paper]{article}
\usepackage{SogangThesis}

%%%%% 문서 설정 %%%%%
%\phdfalse
\name{홍 길 동}
\title{Ph.D Thesis Title Test}
\category{이학}
\department{물 리 학}
\ayear{2018년 12월}
\adate{2018년 OO월 OO일}
\supervisor{지 도 교 수}
\refone{레 프 리 1}
\reftwo{레 프 리 2}
\refthr{레 프 리 3}
\reffou{레 프 리 4}
\reffiv{레 프 리 5}

%\loffalse
%\lotfalse

%\ackfalse
%\acktitle{감사의 글}	
\acknowledgement{
감사합니다.
}

%\abefalse
%\abetitle{Abstract}
\engabstract{
Abstract contents
}
%\abekeyfalse
\engkeywords{Keywords1, Keywords2}

%\abkfalse
%\abktitle{초 록}
\korabstract{
초록 본문
}
%\abkkeyfalse
\korkeywords{키워드1, 키워드2}



%%%%% 본문 시작 %%%%%
\begin{document}
\maketitle
\newpage
\pagenumbering{arabic}


\section{Introduction}
section test

\subsection{subsection}

subsection test

\subsubsection{subsubsection}

subsubsection test

\paragraph{paragraph}

paragraph test

\begin{figure}[htb]
\centering
\includegraphics[width=0.5\textwidth]{sogang}
\caption{SG logo}	% 캡션이 없으면 LOF에 표기되지 않음
\label{fig:sglogo}
\end{figure}


\begin{table}[htb]
\centering
\begin{tabular}{c|c}
test11 & test12 \\
test21 & test22
\end{tabular}
\caption{test table}	% 캡션이 없으면 LOT에 표기되지 않음
\label{tab:test}
\end{table}

\nocite{*}	% 테스트용으로, 반드시 삭제!!!



%%%%% 참고 문헌 %%%%%
\newpage
\setstretch{1.0}
\bibliographystyle{./bib/modified_jhep}
\bibliography{./bib/references}
%\begin{thebibliography}{000}	% 둘 중 하나만 uncomment, BibTeX 추천!!!
%\bibitem{test1}
%  Author 1, 
%  ``title 1,''  
%  journal 1.
%\bibitem{test2} 
%  Author 2, 
%  ``title 2,''  
%  journal 2.
%\bibitem{test3} 
%  Author 3, 
%  ``title 3,''  
%  journal 3.
%\end{thebibliography}
\end{document}
