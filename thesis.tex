% !TEX encoding = UTF-8

\documentclass[11pt,a4paper]{article}			% 출력물 A4 기준 서강대학교 졸업논문 형식 
\usepackage{SogangThesis}

%%%%%%%%%%%%%%%%%%%%%%%%%%%%%%%%%%%%%%%%%%%%%%%%%
%			필수 입력  							%
%%%%%%%%%%%%%%%%%%%%%%%%%%%%%%%%%%%%%%%%%%%%%%%%%
\name{홍 길 동}									% 본인의 이름, 간격 주의
\title{Ph.D Thesis Title} 						% 논문 제목
\category{이학}									% 이학, 공학 등등
\department{물 리 학}							% 본인의 학과, 간격 주의
\ayear{2018년 12월}								% 졸업 연도, 간격 주의
%\degreefalse									% 석사
\adate{2018년 OO월 OO일} 						% 논문 인준 날짜, 간격 주의
\supervisor{지 도 교 수}							% 지도교수 이름, 간격 주의
\refone{레 프 리 1}								% 주심 이름, 간격 주의
\reftwo{레 프 리 2}								% 부심 이름 차례로	4명
\refthr{레 프 리 3}
\reffou{레 프 리 4}								% 석사는 무시
\reffiv{레 프 리 5}							% 석사는 무시

%\loffalse										% 사진 목록 [제거]
%\lotfalse										% 테이블 목록 [제거]

%\ackfalse										% Ack. [제거]
%\acktitle{감사의 글}								% Ack. <제목>
\acknowledgement{								% 감사의 글 (시작)
감사합니다.
}												% 감사의 글 (끝)

%\abefalse										% 영문 초록 [제거]
%\abetitle{Abstract}							% 영문 초록 <제목>
\engabstract{									% 영문 초록 (시작)
Abstract contents
}												% 영문 초록 (끝)
%\abekeyfalse									% 영문 키워드 [제거]
\engkeywords{Keywords1, Keywords2}				% 영문 키워드

%\abkfalse										% 국문 초록 [제거]
%\abktitle{초 록}								% 국문 초록 <제목>
\korabstract{									% 국문 초록 (시작)
초록 본문
}												% 국문 초록 (끝)
%\abkkeyfalse									% 국문 키워드 [제거]
\korkeywords{키워드1, 키워드2}					% 국문 키워드


%%%%%%%%%%%%%%%%%%%%%%%%%%%%%%%%%%%%%%%%%%%%%%%%%
%			본문 시작 							%
%%%%%%%%%%%%%%%%%%%%%%%%%%%%%%%%%%%%%%%%%%%%%%%%%
\begin{document}
\maketitle
\newpage
\pagenumbering{arabic}							% 본문 페이지 번호 형식을 숫자로 설정



\section{Introduction}
section test

\subsection{subsection}

subsection test

\subsubsection{subsubsection}

subsubsection test

\paragraph{paragraph}

paragraph test

\begin{figure}[htb]
\centering
\includegraphics[width=0.5\textwidth]{sogang}
\caption{SG logo}								% 캡션이 없으면 LOF에 표기되지 않음
\label{fig:sglogo}
\end{figure}


\begin{table}[htb]
\centering
\begin{tabular}{c|c}
test11 & test12 \\
test21 & test22
\end{tabular}
\caption{test table}							% 캡션이 없으면 LOT에 표기되지 않음
\label{tab:test}
\end{table}

\nocite{*}										% 테스트용으로, 반드시 삭제!!!



%%%%%%%%%%%%%%%%%%%%%%%%%%%%%%%%%%%%%%%%%%%%%%%%%
%			참고 문헌 							%
%%%%%%%%%%%%%%%%%%%%%%%%%%%%%%%%%%%%%%%%%%%%%%%%%
\newpage										% 새 페이지 
\setstretch{1.0} 								% 참고 문헌의 줄간격 설정 
\bibliographystyle{./bib/modified_jhep}			% 참고 문헌 스타일 파일 
\bibliography{./bib/references}					% 참고 문헌 bib 파일
%\begin{thebibliography}{000}					% 둘 중 하나만 uncomment
%\bibitem{test1} 								% BibTeX 추천!!!
%  Author 1, 
%  ``title 1,''  
%  journal 1.
%\bibitem{test2} 
%  Author 2, 
%  ``title 2,''  
%  journal 2.
%\bibitem{test3} 
%  Author 3, 
%  ``title 3,''  
%  journal 3.
%\end{thebibliography}
\end{document}
